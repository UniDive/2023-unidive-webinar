\usepackage{xcolor}

%\newcommand{\ile}[1] {{\scriptsize \textcolor{RoyalBlue}{\textit{#1}}}}
%\newcommand{\lit}[1]{{\scriptsize \textcolor{RoyalBlue}{`#1'}}} %Literal translation
%\newcommand{\idio}[1]{{\scriptsize \textcolor{RoyalBlue}{`#1'}}}  %Idiomatic translation
%\newcommand{\exlit}[2]{\ile{#1}~\lit{#2}} %Example with a literal translation
%\newcommand{\exidio}[2]{\ile{#1}~\idio{#2}} %Example with an idiomatic translation
%\newcommand{\litidio}[2]{\lit{#1}$~\Rightarrow~$\idio{#2}} %Literal and idiomatic translation
%\newcommand{\exlitidio}[3]{\ile{#1}~\lit{#2}~$\Rightarrow$~\idio{#3}} %Example with a literal and a and idiomatic translation

\newcommand{\ana}[1] {{\scriptsize \textcolor{RoyalBlue}{#1}}} %Output of an analysis of a text 
\newcommand{\iletrue}[1] {\textcolor{OliveGreen}{#1}}
\newcommand{\ilefalse}[1] {\textcolor{Red}{#1}}
\newcommand{\ileunsure}[1] {\textcolor{BurntOrange}{#1}}
%\newcommand{\lex}[1] {\textbf{#1}} %Lexicalized component


%%%%%%%%%%%
%% in-line examples for languages in Latin script
%%%%%%%%%%
\usepackage{ulem}
\newcommand{\lex}[1]{\textbf{#1}}  %Lexicalized component
\newcommand{\litlex}[1]{\uwave{#1}}  %Literal reading of a lexicalized component
\newcommand{\coinlex}[1]{\dashuline{#1}}  %Coincindental co-occurrence of lexicalized components

\newcommand{\ile}[1]{\textcolor{blue}{\textsl{#1}}} %In-line example  
\newcommand{\lit}[1]{\textcolor{gray}{`#1'}} %Literal translation
\newcommand{\idio}[1]{\textcolor{brown}{`#1'}}  %Idiomatic translation
\newcommand{\exlit}[2]{\ile{#1}~\lit{#2}} %Example with a literal translation
\newcommand{\exidio}[2]{\ile{#1}~\idio{#2}} %Example with an idiomatic translation
\newcommand{\litidio}[2]{\lit{#1}$ \Rightarrow $\idio{#2}} %Literal and idiomatic translation
\newcommand{\exlitidio}[3]{\ile{#1}~\lit{#2}$\Rightarrow$\idio{#3}} %Example with a literal and a and idiomatic translation


%%%%%%%%%%
\setbeamertemplate{itemize/enumerate subbody begin}{\scriptsize}

\usepackage{relsize}
\newcommand{\lang}[1]{\textcolor{gray}{\textsmaller[1.5]{\fbox{\textsf{#1}}}}}

\newcommand{\formatPOS}[1]{{\scriptsize #1}}
\newcommand{\formatMorph}[1]{\_}  %{{\tiny #1}}
\newcommand{\formatMWE}[1]{\textcolor{blue}{#1}}

\newcommand*\GBitem{%
   \item[{\includegraphics[width=0.2cm]{Images/Flags/UnitedKingdom.png}}]}
\newcommand*\FRitem{%
   \item[{\includegraphics[width=0.2cm]{Images/Flags/France.png}}]}
\newcommand*\PLitem{%
   \item[{\includegraphics[width=0.2cm]{Images/Flags/Poland.png}}]}
\newcommand*\SRitem{%
   \item[{\includegraphics[width=0.2cm]{Images/Flags/Serbia.png}}]}
\newcommand*\ITitem{%
   \item[{\includegraphics[width=0.2cm]{Images/Flags/Italy.png}}]}

\newcommand{\gloss}[1] {
  {\scriptsize \textcolor{blue}{'#1'}}}

%%%%% Linguistic examples in blue scriptsize italic
\newcommand{\exa}[1]{\textcolor{blue}{\textit{#1}}}

%%%%% Bibliographic references in scriptsize
\newcommand{\mycite}[1]{{\scriptsize \textcolor{magenta}{\cite{#1}}}}
\newcommand{\minicite}[1]{{\scriptsize \cite{#1}}}
%\newcommand{\minicitet}[1]{{\scriptsize \citet{#1}}}

%%%%% Bets results
\newcommand{\best}[1]{\textbf{#1}}

